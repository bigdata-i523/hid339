\documentclass[sigconf]{acmart}

\usepackage{hyperref}

\usepackage{endfloat}
\renewcommand{\efloatseparator}{\mbox{}} % no new page between figures

\usepackage{booktabs} % For formal tables

\settopmatter{printacmref=false} % Removes citation information below abstract
\renewcommand\footnotetextcopyrightpermission[1]{} % removes footnote with conference information in first column
\pagestyle{plain} % removes running headers

\begin{document}
\title{Big Data Application in Finance}


\author{Hady Sylla}
\affiliation{%
  \institution{Indiana University}
  \streetaddress{Smith Research Center}
  \city{Bloomington} 
  \state{IN} 
  \postcode{47408}
  \country{USA}}
\email{hsylla@iu.edu}


% The default list of authors is too long for headers}
\renewcommand{\shortauthors}{G. v. Laszewski}


\begin{abstract}
This paper is about Big Data application in the financial industries.The paper explore contribution that big data had on the analysis of numerous data by researchers
\end{abstract}

\keywords{HID 339, Big Data, Breast Cancer, Cancer}



\maketitle


\section{Introduction}

During the 1980s and 1990s, information technology (IT) systems “transformed virtually every single bank process, noted McKinsey \& Co. director Toos Daruvala. He wasn’t too far off – automated teller machines, or ATMs not only expanded, but were also connected to other banks, so those who wanted money could withdraw from an ATM machine that wasn’t owned by their particular bank (for a fee, of course). Data became better automated, and early software programs allowed businesses to connect their payrolls to their banks.
Daruvala wasn’t finished with his assessment. He indicated that these days, “banks have that rare opportunity to reinvent themselves again – with data and analytics.” Basically, Daruvala spoke of banks and other financial institutions being able to differentiate themselves from the competition with help from Big Data. Any decision made by these institutions, from driving revenue, to controlling costs, to mitigating risk would be made with a lot of data and analytics \cite{daruvala2013advanced}. Daruvala also noted that Big data is hugely important in the financial industry, especially given the large number of regulations under which most banks, lenders and other institutions operate. 
Daruvala isn’t wrong in his statements. Big Data is helping to change the way in which financial institutions operate, what types of products they develop, and how they interact with consumers. But what, exactly, is meant by Big Data? How are financial institutions embracing the technology? And even more important, what are some of the applications that are being driven by Big Data.
 

\section{Big Data Application for Breast Cancer Treatment}

\setlength{\parskip}{1em}
\renewcommand{\baselinestretch}{2.0}

\section{Defining “Big Data”}

What, exactly is meant by “Big Data?” This is a term that seems to be thrown around by a variety of industries, but there doesn’t seem to be a specific, or single definition. \cite{peat2013} points out that Big Data is a somewhat broad term; as he puts it, “a vague concept without a common definition” (p. 10). From the technical side, Big Data involves a data set of huge samples that typically can’t be handled by the standard database management systems, and that require specialized technology to store and process \cite{peat2013}. That technology includes distributed storage systems, data mining and cloud computing, which, in turn, encompasses automatic classification, processing, effective storage and parallel computing \cite{peat2013}.
Big Data is also known as data analytics, which, in turn, is responsible for linking a bank’s internal data – such as customer accounts, payment histories, credit scoring and assets, to external data – which is focused on interest rates, macroeconomic trends, and market/customer preferences (O’Neil, 2016). 
For purposes of this paper, we’ll define Big Data as information collected and used by entities and institutions to help drive business strategies. And, one additional aspect of Big Data that needs to be defined is a growing trend toward a concept known as “financial technology;” or “fintech,” for short.

\section{The Concept of Fintech}

To understand Big Data applications in finance, it’s important to understand where fintech plays a role. Fintech originally applied to back-end operations of consumer and trade financial institution \cite{he2017fintech}. These days, the term encompasses technological innovations and uses \cite{he2017fintech}. At one time, banks and institutions used the market’s “invisible hand” to make financial decisions. These days, predictive behavioral analytics and data-driven marketing are starting to take the guesswork out of financial decisions \cite{he2017fintech}. Apps are helping users boost their spending and savings decisions \cite{he2017fintech}. And, improved data analytics are assisting institutional clients in refining investing decisions, as well as to open different opportunities for financial applications.

\section{Applying Big Data in Finance}

In the finance industry, plenty of data are created on a daily basis; this can include prices, transactions, credit records, customer information and so on \cite{peat2013}. Furthermore, financial institutions are taking advantage of the Big Data technology for better strategizing, the ability to offer better services, and to obtain a higher degree of market share \cite{peat2013}. Breaking this down, Big Data can be used for continuous risk management (predicting the financial institution’s performance, and to mitigate risk, when necessary); accurate marketing (understanding everything about the customer, from location, to credit records, to consumption habits); effective operations/management (responses to policy and other aspects) and instant response to the market \cite{peat2013}.
That’s the technical side, the “jargon” side of what big data and banking are about. Big Data is very helpful in areas such as compliance – most banks in the United States, for example,  operate under anti-money laundering regulations, as well as Foreign Account Tax Compliance Act (FATCA), Foreign Corrupt Practices Act (FCPA), and edicts passed down by the Financial Industry Regulatory Authority, or FINRA, at least in the United States \cite{gabor2017digital}. 
Big Data also helps banks to price their products effectively; thanks to customer and market information, the banks can steer away from mass marketing strategies and limited response rates, and focus on smaller campaigns directed toward fewer, but more responsive, customers \cite{gabor2017digital}. And, as alluded to above, Big Data can also present a stronger picture of risks and performance, thus improving due diligence when it comes to everything from mergers and acquisitions, to approving a home loan to a first-time buyer \cite{gabor2017digital}.
Daruvala provides three examples of banks who relied on Big Data to improve their market share and risk-management areas.
One bank, which attempted to refresh its small-business underwriting model, relied on third-party data from external sources, and applied analytic techniques to redo their models, using the Gini coefficient. The Gini coefficient measures a model’s ability to tell the difference between good risks and bad risks. The bank’s new model ended up improving loans to small businesses.
Another institution, working in developing markets, opted to obtain data from the local telephone companies; “the paying behavior for the telco is actually a great predictive indicator for the credit behavior with the bank,” Daruvala observes. The bank acquired the telco data, joined it with its own in-house data, and ended up with a large improvement in underwriting.
Finally, a third institution relied on Big Data for marketing purposes, by adding external data around social media to determine the right next product to offer consumers. In that way, the institution was able to make the right offers to consumers whenever they walked into a branch or called into the customer service center. The predictor model worked very well, with the right products being targeted to the right consumers.
Other applications can study the use of customer checking accounts, automatic deposit and ACH information to determine if a customer has a new job, or could be eligible for a higher-return money market fund \cite{he2017fintech}. Big Data can help banks reach out to consumers who might be more willing to engage with them. “There’s nothing worse than being approached when you don’t want to be,” said Ed O’Brien, director of the banking channels practice at research firm Mercator Advisory Group \cite{he2017fintech}. He added that, with Big Data, banks would have a chance of reaching out to whom they want, when they want, and that the results would be relevant to them \cite{he2017fintech}.



\section{Conclusion}

There are many ways in which Big Data can be used by banks and other financial institutions. On the operations side, such data can be pulled to determine risk management and to ensure regulatory compliance. On the marketing side, the information is useful for building customer predictor models, and to determine what products might be most appealing to a certain target audience. Cost containment, mobile technology and fraud prevention are also good uses for such institutions.
In short, Big Data has multiple uses for financial institutions. As the technology continues to improve, so will the ways in which banks use the information they have acquired, and stored.



\begin{acks}

  I would like to thank Dr. Gregor von Laszewski for his
  support and suggestions to write this paper.

\end{acks}

\bibliographystyle{ACM-Reference-Format}
\bibliography{report}


\end{document}
